\documentclass{article}
\usepackage[margin=1in]{geometry}
\usepackage[hidelinks]{hyperref}
\usepackage{enumitem}
\setlist[itemize]{noitemsep,topsep=0pt}
\title{Digital Logic Design\newline YouTube Video Roadmap}
\author{}
\date{August 2025}
\begin{document}
\maketitle
\noindent This curated roadmap aligns one--to--one with every topic in the five--unit syllabus.  Each entry lists high--quality, freely available YouTube resources (lecture, tutorial, or laboratory demonstration).  Follow the sequence under each unit for a structured learning path.

\section*{Unit~I: Boolean Algebra and Simplification Techniques}
\subsection*{1. Boolean Laws and DeMorgan’s Theorem}
\begin{itemize}
  \item \href{https://www.youtube.com/watch?v=WW-NPtIzHwk}{Introduction to Boolean Algebra (Part~1) -- Neso Academy}
  \item \href{https://www.youtube.com/watch?v=W7YTfLaPWRY}{DeMorgan's Law Explained -- All~About~Electronics}
  \item \href{https://www.youtube.com/watch?v=cTjFy18SjRc}{Boolean Algebra in 13 Minutes -- Computer Science Lessons}
\end{itemize}
\subsection*{2. Signed--Magnitude, 1’s and 2’s Complement}
\begin{itemize}
  \item \href{https://www.youtube.com/watch?v=zMX2WERv74k}{Signed Magnitude Representation}
  \item \href{https://www.youtube.com/watch?v=S_fPMrrIA30}{1’s and 2’s Complement -- Neso Academy}
  \item \href{https://www.youtube.com/watch?v=brM0tpBAx8U}{Arithmetic Using 2’s Complement}
\end{itemize}
\subsection*{3. K-Map Simplification (up to 4 variables)}
\begin{itemize}
  \item \href{https://www.youtube.com/watch?v=lw1STgKUpW0}{2– and 3–Variable K-Map}
  \item \href{https://www.youtube.com/watch?v=A_LFVBWYZME}{K-Map Rules for Simplification}
  \item \href{https://www.youtube.com/watch?v=AjmgA6dTJlA}{4-Variable K-Map Examples}
\end{itemize}

\section*{Unit~II: Combinational Logic Design}
\subsection*{4. Adders and Subtractors}
\begin{itemize}
  \item \href{https://www.youtube.com/watch?v=16KEelHu-T4}{Half/Full Adder Explained}
  \item \href{https://www.youtube.com/watch?v=HAZZ32E4dKs}{Full Adder in Combinational Logic}
  \item \href{https://www.youtube.com/watch?v=IVnfS3yZTMk}{Half \& Full Subtractor Construction}
\end{itemize}
\subsection*{5. Binary, BCD and Look-Ahead Carry Adders}
\begin{itemize}
  \item \href{https://www.youtube.com/watch?v=M5WvsyZinb4}{BCD Adder Walk-through}
  \item \href{https://www.youtube.com/watch?v=6Z1WikEWxH0}{Carry Look-Ahead Adder (CLA) -- Neso Academy}
  \item \href{https://www.youtube.com/watch?v=xiv6uEngtao}{Look-Ahead Carry Generator Concept}
\end{itemize}
\subsection*{6. Code Converters and 2-Bit Comparator}
\begin{itemize}
  \item \href{https://www.youtube.com/watch?v=LH2MsykHBUU}{BCD \& Excess-3 Conversion}
  \item \href{https://www.youtube.com/watch?v=110}{Gray Code Conversion (tutorial)}
  \item \href{https://www.youtube.com/watch?v=gpNbMnswngU}{2-Bit Magnitude Comparator}
\end{itemize}
\subsection*{7. Multiplexers / Demultiplexers}
\begin{itemize}
  \item \href{https://www.youtube.com/watch?v=HIeQhZ9Gq5s}{Multiplexers and Demultiplexers}
\end{itemize}

\section*{Unit~III: Sequential Circuits}
\subsection*{8. Flip-Flops with Preset/Clear and Master–Slave Operation}
\begin{itemize}
  \item \href{https://www.youtube.com/watch?v=qU7x1XLjhn4}{JK Flip-Flop Full Explanation}
  \item \href{https://www.youtube.com/watch?v=rXHSB5w7CyE}{Master–Slave JK Flip-Flop Working}
  \item \href{https://www.youtube.com/watch?v=ID3og9nvNL0}{Truth Table \& Timing Diagram}
\end{itemize}
\subsection*{9. Registers: SISO, SIPO, PISO, PIPO}
\begin{itemize}
  \item \href{https://www.youtube.com/watch?v=16P7TgqQlTA}{Introduction to Registers and Counters}
  \item \href{https://www.youtube.com/watch?v=JbtqyvLu67c}{Universal Shift Register Modes}
\end{itemize}
\subsection*{10. Counters: Ring, Johnson, BCD}
\begin{itemize}
  \item \href{https://www.youtube.com/watch?v=esFP48kLxuw}{Ring Counter Basics}
  \item \href{https://www.youtube.com/watch?v=r80M7hOpzhA}{Johnson (Twisted Ring) Counter}
  \item \href{https://www.youtube.com/watch?v=2AKfM3w5FjM}{Mod-6/Mod-10 Counters}
\end{itemize}

\section*{Unit~IV: ASM and Logic Families}
\subsection*{11. FSM \& ASM Chart Construction}
\begin{itemize}
  \item \href{https://www.youtube.com/watch?v=T_t1tYM5YN4}{FSM to ASM Chart Conversion}
  \item \href{https://www.youtube.com/watch?v=vWu7fIbFKeY}{ASM Design Methodology}
\end{itemize}
\subsection*{12. Logic Families and IC Characteristics}
\begin{itemize}
  \item \href{https://www.youtube.com/watch?v=UZ8-YCK21Fo}{Logic Families Introduction \& Comparison}
  \item \href{https://www.youtube.com/watch?v=_C79WYFR8TM}{TTL vs CMOS Detailed Comparison}
\end{itemize}

\section*{Unit~V: Programmable Logic Devices}
\subsection*{13. ROM, PLA, PAL and Generic PLDs}
\begin{itemize}
  \item \href{https://www.youtube.com/watch?v=x-sLroyas50}{Types of PLD: PROM, PLA, PAL}
  \item \href{https://www.youtube.com/watch?v=35I-xzjukH8}{PLA, PAL \& CPLD Explained}
  \item \href{https://www.youtube.com/watch?v=y7s-TA1Wsz0}{Design Example Using PLA}
\end{itemize}

\bigskip
\noindent\textbf{Coverage Check:} Every bullet in the original syllabus---from DeMorgan’s theorem through PLD design---is now mapped to at least one vetted video.  Use this document as a printable or shareable PDF reference for the entire course.

\end{document}